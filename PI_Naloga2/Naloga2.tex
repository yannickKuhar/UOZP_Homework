\documentclass[11pt]{article}

\usepackage[utf8]{inputenc}

% header
\makeatletter
\def\@maketitle{%
  \noindent
  \begin{minipage}{2in}
  \@author
  \end{minipage}
  \hfill
  \begin{minipage}{1.2in}
  \textbf{\@title}
  \end{minipage}
  \hfill
  \begin{minipage}{1.2in}
  \@date
  \end{minipage}
  \par
  \vskip 1.5em}
\makeatother

% Naloga
\title{Naloga 2}
% Ime Priimek (vpisna)
\author{Yannick Kuhar (63160187)}
\date{\today}

\begin{document}

\maketitle

\section{Izbrani jeziki.}

L = jezik je v Latinici \\
C = jezik je v Cirilici \\
D = drugo

\begin{center}
\begin{tabular}{1111111}
 \textbf{Slovanski jeziki:}& Slv(L) & Slo(L) & Rus(C) & Blg(C) & Czc(L) & \\
 \textbf{Germanski jeziki:} & Eng(L) & Ger(L) & Swd(L) & Fin(L) & Ice(L) & \\
 \textbf{Romanski jeziki:}& Fra(L) & Esp(L) & Por(L) & Ita(L) & \\
 \textbf{Ostalo:} & Swa(L) & Gre(D) & Trk(L) & Kkn(D) & Chn(D) & Jpn(D) & \\
\end{tabular}
\end{center}

\subsection{Predobdelava datotek.}
Sprva smo sestavili slovar, oblike \{'država': besedilo\}, katerega smo na predavanjih imenovali corpus.
Nato smo iz besedila sestavili vse možne trojice znakov besedila in izračunali njihovo frekvenco. Tako smo dobili točke v večdimenzionalnem prostoru, oblike \{'država': tocka(vektor frekvenc)\}.

\section{Rezultati razvrščanja.}
V tem razdelku si bomo pomagali s tremi priloženimi datotekami (histogram.txt, minSihueta.txt in maxSihueta.txt), v katerih se nahajajo rezultati. Podobnosti sihuet se nahajajo na intervalu [0.21, 0.4], kar kaže, da imajo jeiki skupnega prednika. Na podoben, čeprav manj učinkovit način, se naredijo skupine.

\section{Napovedovanje jezika.}
Uporabimo podatke, ki jih že imamo, torej slovar točk. Besedilo predelamo na isti način kot vhodne podatke in dobimo točko, nato iteriramo čez slovar točk in v vsaki iteraciji izračunamo kosinusno razdaljo med točkama. Jezik določimo na podlagi maksimalne kosinusne razdalje.

\subsection{Tabela.}
\begin{left}
\begin{tabular}{1111111}
	Danes je lep dan, na nebu ni bilo nobenega oblaka. & SLOVAK & Napačno & \\
	Today is a good day, there are no clouds in the sky. & SLV & Napačno &\\
	Heute ist ein guter Tag es gibt keine Wolken an dem Himmel.& GER & Pravilno & \\
	Dnes je dobrý den, na obloze nejsou žádné mraky. & ITN & Napačno & \\
	Aujourd'hui est un bon jour, il n'y a pas de nuages ​​dans le ciel. & FIN & Napačno &
\end{tabular}
\end{left}

Glede na rezultate metoda ni natančna.

\section{Bonus naloga: Članki.}
V tem razdelku bomo primerjali rezultate med naborom prevodov Deklaracije o človekovih pravicah (rezultati v histogram.txt) in naborom 20-ih člankov iz interneta (histogram\_clankov.txt). Glavna razlika med histogramoma je, da je histogram prevodov porazdeljen na intervalu [0.21, 0.4], medtem ko je histogram člankov porazdeljen na intervalu [-0.85, 0.35]. Sklepam, da je razlika med intervaloma zaradi drastične razlike v vsebini dokumentov.

\end{document}












