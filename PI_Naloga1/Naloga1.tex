
\documentclass[11pt]{article}

\usepackage[pdftex]{graphicx}

% header
\makeatletter
\def\@maketitle{%
  \noindent
  \begin{minipage}{2in}
  \@author
  \end{minipage}
  \hfill
  \begin{minipage}{1.2in}
  \textbf{\@title}
  \end{minipage}
  \hfill
  \begin{minipage}{1.2in}
  \@date
  \end{minipage}
  \par
  \vskip 1.5em}
\makeatother

% Naloga
\title{Naloga 1}
% Ime Priimek (vpisna)
\author{Yannick Kuhar (63160187)}
\date{\today}

\begin{document}

\maketitle

% \section{Uvod.}
% Cilj naloge Glasovanje za Pesem Evrovizije je bil narediti uvod v podatkovno rudarjenje.  % To smo dosegli s programiranjem osnovne metode Hierarchical clustering v programskem %jeziku Python. Cilj je bil tudi spoznati osnove  Pythona katerega še nisem uporabljal.

\section{Podatki.}
Dobili smo podatke o državah tekmovalkah na Evroviziji od leta 1998 do leta 2009 in sicer 291 primerov, vsakega s 63 atributi. V nalogi je bilo bistveno analizirati glasovanja med državami, kar pomeni, da je bilo nekaj nezanimivih atributov kot so ime avtorja, naslov pesmi, jezik petja in nato še nekaj podatkov o avtorju in državi gostiteljici - bolj natančno atributi od 3. do 16. stolpca.

\subsection{Interpretacija in čiščenje podatkov.}
Ugotoviti smo morali pristranskost glasovanja držav, zato sem seštel vse glasove skozi leta in sestavil slovar oblike \{('država'): ['1.0', '2.0', ... '0.0']\} in ga nato transponiral. Manjša težava so bili le pomanjkljivi podatki zaradi domnevno nesodelovanja. Manjkajoča polja sem nadomestil z '0.0', saj skoraj ne vpliva na končen rezutat.

\section{Implementacija algoritma.}
Skupine sem združeval na podlagi Evklidske razdalje, skupke skupin pa po razdalji med povprečjem vseh primerov (angl. avg. linkage), ker sem domneval, da je najenostavnejša za implementirati v Pythonu.

\section{Rezultati.}

Zaradi velikosti dendrograma sem ga vključil v datoteko Drevo.txt.

\subsection{Skupine.}
\textbf{Skupina 01:} Croatia, Slovenia, Bosnia and Herzegovina, Macedonia, Montenegro\\
\textbf{Skupina 02:} Greece, Russia, Poland, Andorra, Portugal, Armenia, Belarus, Hungary, Czech Republic, Moldova, Serbia, Monaco, Azerbaijan, San Marino, Slovakia, Romania, Albania, Bulgaria, Spain, Cyprus, Israel   
\textbf{Skupina 03:} Estonia, Latvia, Lithuania, Ireland, Malta, Iceland, Norway, Denmark, Finland \\
\textbf{Skupina 04:} France, Germany, Belgium, Netherlands

\begin{tabular}{ c c c }
 Skupina & Preferirana država & Nepreferirana država \\ 
 \textbf{Skupina 01:} & Czech Republic & Azerbaijan	 \\  
 \textbf{Skupina 02:} & Bolgaria & Bosnia and Herzegovina\\
 \textbf{Skupina 03:} & Cyprus & Croatia \\
 \textbf{Skupina 04:} & Hungary & Albania   
\end{tabular}

\subsection{Razlaga.}
Skupine sem izbral na podlagi dendrograma, kot je profesor povedal na predavanjih sem uporabil navidezno črto, ki je razdelila dendrogram na 5 skupin. Preferirane države sem izbral po motodi prefinnepref(hc).

\end{document}