\documentclass[11pt]{article}

\usepackage[pdftex]{graphicx}

% header
\makeatletter
\def\@maketitle{%
  \noindent
  \begin{minipage}{2in}
  \@author
  \end{minipage}
  \hfill
  \begin{minipage}{1.2in}
  \textbf{\@title}
  \end{minipage}
  \hfill
  \begin{minipage}{1.2in}
  \@date
  \end{minipage}
  \par
  \vskip 1.5em}
\makeatother

% Naloga
\title{Naloga 3}
% Ime Priimek (vpisna)
\author{Yannick Kuhar (63160187)}
\date{\today}

\begin{document}

\maketitle

\section{Uvod}
Cilj naloge je bil, da ustvarimo čim bolj natančni model za napovedovanje prihoda LPP mestnih avtobusov. To sem dosegel z obdelavo vhodnih podatkov, ki bo opisana v sledečih odstavkih. Nad obdelanimi sem nato sestavil napovedni model s pomočjo linearne regresije, ki so jo ponudili izvajalci predmeta.

\section{Ocenjevanje tocnosti}
Točnost sem ocenjeval z lastno implementacijo MSE(mean square error), ki mi je za primer \textbf{LinearnaRegresija}(izvedena nad razlikam v minutah) vrnila: MSE(LinearnaRegresija) = 4.386 \\
Torej v povprečju se zmoti za 4 minute in 23 sekund, kar je še vseeno, po mojih laičnih opažanjih bolje kot trenutni LPP-jev sistem.

\section{Napovedni modeli}
\begin{description}
\item[LinearnaRegresija]
Iz danih podatkov sem sestavil točko oblike [idLinije, leto, mesec, dan, indexDneva, aliJeVikend, ura, minuta, sekunda], za obdelavo sem uporabil knjižnico datetime, iz katere sem dobil atribut idxDneva(je na intervalu [0, 6]) in nato določil ali je vikend ali ne. Knjižnico sem uporabil tudi za izvedbo aritmetičnih operacij nad datetime objekti. 
\end{description}

\section{Rezultati}
\begin{tabular}{ c c c }
 Ime & Napaka \\ 
 \textbf{LinearnaRegresija*} & 539.21270 &
\end{tabular}

\subsection{Opis rezultata}Iz rezultata strežnika vidimo, da model ni natančen. Domnevam, da zaradi slabe izbire atributov(posamezni atributi npr. idLinije doprinese premalo) oz. premajhnega števila atributov, želel sem dodati še podatke o vremenu(temperatura, vlaga, količina dežja/snega) in geografsko razdaljo med začetno in končno postajo vendar zaradi časovne stiske nisem utegnil. Iz istega razloga sem izdelal le en model.
\end{document}
